\begin{chapter}{Geometric and Classification Problems}
    ~\cite{EE364a-extra}\textbf{Exercise 8.8}. \textit{Bounding object position from multiple camera views}.
$x \in \mathbf{R}^3$, and viewed by a set of $m$ cameras. Our goal is to find a box in $\mathbf{R}^3$,
\[
\mathcal{B}=\left\{z \in \mathbf{R}^3 \mid l \preceq z \preceq u\right\},
\]
for which we can guarantee $x \in \mathcal{B}$. We want the smallest possible such bounding box. (Although it doesn't matter, we can use volume to judge 'smallest' among boxes.)

\noindent Now we describe the cameras. The object at location $x \in \mathbf{R}^3$ creates an image on the image plane of camera $i$ at location
\[
v_i=\frac{1}{c_i^T x+d_i}\left(A_i x+b_i\right) \in \mathbf{R}^2 .
\]

\noindent The matrices $A_i \in \mathbf{R}^{2 \times 3}$, vectors $b_i \in \mathbf{R}^2$ and $c_i \in \mathbf{R}^3$, and real numbers $d_i \in \mathbf{R}$ are known, and depend on the camera positions and orientations. We assume that $c_i^T x+d_i>0$. The $3 \times 4$ matrix
\[
P_i=\left[\begin{array}{ll}
A_i & b_i \\
c_i^T & d_i
\end{array}\right]
\]
is called the camera matrix (for camera $i$ ). It is often (but not always) the case the that the first 3 columns of $P_i$ (i.e., $A_i$ stacked above $c_i^T$ ) form an orthogonal matrix, in which case the camera is called orthographic.
We do not have direct access to the image point $v_i$; we only know the (square) pixel that it lies in. In other words, the camera gives us a measurement $\hat{v}_i$ (the center of the pixel that the image point lies in); we are guaranteed that
\[
\left\|v_i-\hat{v}_i\right\|_{\infty} \leq \rho_i / 2,
\]
where $\rho_i$ is the pixel width (and height) of camera $i$. (We know nothing else about $v_i$; it could be any point in this pixel.)
Given the data $A_i, b_i, c_i, d_i, \hat{v}_i, \rho_i$, we are to find the smallest box $\mathcal{B}$ (i.e., find the vectors $l$ and $u$ ) that is guaranteed to contain $x$.
In other words, find the smallest box in $\mathbf{R}^3$ that contains all points consistent with the observations from the camera.

\vspace{0.1cm}
\noindent (a) Explain how to solve this using convex or quasiconvex optimization. You must explain any transformations you use, any new variables you introduce, etc.
If the convexity or quasiconvexity of any function in your formulation isn't obvious, be sure justify it.

\vspace{0.1cm}
\noindent (b) Solve the specific problem instance given in the file \lstinline|camera_data.m|. Be sure that your final numerical answer (i.e., $l$ and $u$ ) stands out.

\vspace{0.1cm}
\noindent \textbf{Response.} Let's summarize the prompt and understand the problem .

$x$ is some object in the physical world (modeled as 3-dimensional Euclidean space). We have $m$ cameras which capture images
of $x$. However, because

Let's first formulate the constraints ensuring that our proposed box that $x$ lies in is consistent with the camera measurements;
\textit{i.e.}, there isn't a point in the bounding box which contra

\[\left\lVert v_i - \hat{v}_i \right\rVert_{\infty} \le \rho_i / 2, , \quad i = 1, \ldots, m\]
is expanded as 
\[\left\lVert \frac{1}{c_i^T x+d_i}\left(A_i x+b_i\right) - \hat{v}_i \right\rVert_{\infty} \le \rho_i/2, \quad i = 1, \ldots, m.\]
The $\ell_{\infty}$-norm is requiring that the absolute value of both terms in the $\mathbf{R}^2$ vector
\[\frac{1}{c_i^T x+d_i}\left(A_i x+b_i\right) - \hat{v}_i\]
be less than or qual to $\rho_i/2$. Furthermore,

\[-(\rho_i/2) \bm{1} \preceq \frac{1}{c_i^T x+d_i}\left(A_i x+b_i\right) - \hat{v}_i \preceq (\rho_i/2) \bm{1}, \quad i = 1, \ldots, m\]

\[(\hat{v}_i - (\rho_i/2) \bm{1})(c_i^T x + d_i) \preceq A_i x + b_i \preceq (\hat{v}_i + (\rho_i/2))(c_i^Tx + d_i), \quad i = 1, \ldots, m\]

\end{chapter}