\begin{chapter}{Approximation and Fitting}

    % norm approximation => a way of measuring closeness

    % Signal Reconstruction (~15m mark Lec 10 EE364a)
    % principled way to choose amount of smoothing: cross-validation
    % how do to it:
        % take x corrupted, and extend formulation to allow it to handle NANs (/ NAs / missing value) <- supposedly very simple
        % x hat is going to be a full signal
        % suppose 10% is missing. Randomly yank out another 10% of data. Now do signal reconstruction
        % every time you reconstruct, you make a prediction about the values which you did observe, but that you yanked out of training
        % 

    First use that

    \[\text{minimize} \; f_0(x) = \left\lVert Ax - b  \right\rVert_{2}\]
    is equivalent to (since $f_0$ is nonnegative) 

    \[\text{minimize} \; f_0(x)^2  = \left\lVert Ax - b  \right\rVert_{2}^2, \]
    where $f_0(x) = \left\lVert Ax - b \right\rVert_{2}^2 = \left(Ax-b\right)^T\left(Ax-b\right)$. Important observations
    \begin{itemize}
        \item $f_0$ is convex.
        \item The problem is unconstrained.
    \end{itemize}
    Furthermore, we know that the globally optimal solution, $x^*$, satisfies $\nabla f_0(x^*) = 0$. We use this as an opportunity
    to practice matrix calculus. Specifically, we will use differentials to derive $\nabla f_0$. Start by writing out

    \[df_0 = d \left( \left(Ax-b\right)^T\left(Ax-b\right) \right) = \left(A(x+dx)-b\right)^T\left(A(x+dx)-b\right) - \left(Ax-b\right)^T\left(Ax-b\right). \]
    Recall that we wish to transform this equation into the form $df_0 = Df_0(x)dx$. Or rather,
    we want to re-arrange the expression for $df_0$ so that the $dx$ terms are collected in one place and there's a clear expression for $Df_0(x)$. 
    (Remember that depending on the complexity of the function, it's not always the case that the $dx$s will be collected on the right, as implied above.)
    Of course, because we've already developed matrix calculus
    atoms and compositions, we can take a shortcut. Rewrite $f_0(x) = g(h(x))$ where $g: \mathbf{R}^m \to \mathbf{R}$ and $h: \mathbf{R}^n \to \mathbf{R}^m$ are defined as
    \[g(x) = x^T x \quad \text{and} \quad h(x) = Ax - b,\]
    with corresponding matrix calculus atoms
    \[Dg(x) = 2x^T \quad \text{and} \quad Dh(x) = A.\]
    Also observe what this implies about $f_0$. Instead of being a mapping from $n$-dimensional Euclidean
    space to the real line, it is actually the composition mapping $f_0: \mathbf{R}^n \to \mathbf{R}^m \to \mathbf{R}$.
    This isn't profound, but it can be helpful to keep in mind when performing a derivation such as this one
    since it reinforces the danger of absentmindedly commuting terms. Employing the chain rule, $Df_0(x) = Dg(h(x)Dh(x))$,
    \[Df_0(x) = 2(Ax - b)^T A, \quad \text{so} \quad \nabla f_0(x) = \left(2(Ax - b)^T A \right)^T = 2A^T(Ax-b).\]
    % In your optimality conditions section, specifically say that you are no longer going to walk through the steps
    % of what needs to be done to find the optimal solution of an unconstrained convex problem => it is assumed.
    Equating $\nabla f_0(x) = 0$, we arrive at the normal equations 
    \[A^TAx = A^Tb,\]
    with associated solution
    \[x^* = (A^TA)^{-1} A^T b.\]

\section{Forecasting}
% Low rank forecasting: https://web.stanford.edu/~boyd/papers/pdf/low_rank_forecasting.pdf
% PyTorch forecasting: https://medium.com/@mnitin3/pytorch-forecasting-introduction-to-time-series-forecasting-706cbc48768
% multi variate time series forescasting with LTSM using Python: https://www.youtube.com/watch?v=ODEGJ_kh2aA
    Sources:
    \begin{itemize}
        \item AR Model (page 28 VMLS)
        \item RMS Prediction Error (page 50 VMLS)
        \item Dirichlet energy page 66 VMLS
        \item Finance example page 67 VMLS
        \item Time series auto-correlation page 67
        \item back-test timing page 127
        \item Time series smoothing page 138 VMLS (convolution)
        \item Markov model page 164
        \item Regularizing time series (also Laplacian regularization) page 317 VMLS
        \item Estimating a periodic time series page 318 VMLS (example of regularization)
        \item Example of periodic time series to work on page 15.10 VMLS
        \item de meaned return time series page 378 VMLS
        \item feature matrix page 257 VMLS
    \end{itemize}

    \section{Least Squares Data Fitting}
    \begin{itemize}
        \item ``Least squares is widely used to construct a mathematical model from some data, experiments, or observations.''
        \item Note how least squares is approached before fitting to data.
        \item How to relate to a statistical approach
        \item Is least squares part of the approximation and fitting chapter?
    \end{itemize}

    \section{Recurrent Neural Networks}
    Source: Intro Statistical Learning with Python
    \begin{itemize}
        \item Used when data arise as sequences (so we are considering time series data)
        \item RNNs build models that take into account this sequential nature of the data and build a memory of the past
        \item The feature for each observation is a sequence of vectors
            \[x_t \in \mathbf{R}^n, \quad t=1, \ldots, L\]
    \end{itemize}

    
\end{chapter}